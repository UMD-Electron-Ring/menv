\documentclass[../menv_main.tex]{subfiles}
 
\newcommand\tab[1][1.5cm]{\hspace*{#1}}

\begin{document}

\section{MENV from the command line}

A simple example, in which a .spt file is loaded into MENV and run from the command line (modified from run\_clemnv.m):

\begin{lstlisting}[language=Matlab]
% -- initiate clmenv instance
clm = clmenv();

% -- load .spt file
sptfilename = which('FODO_0_7mA_dipl.spt');
clm.open(sptfilename)

% -- solve
clm.solve()

% -- find periodic solution
clm.periodicmatcher()
\end{lstlisting}


A slightly more complex example, in which a lattice is constructed in MENV from the command line (modified from run\_clemnv\_inj.m):


\begin{lstlisting}[language=Matlab]
clm = clmenv();

% -- set params 

% Beam parameters
emittance = 30; % mm-mrad
perveance = 0.0001045;
x0        = .3; % cm
y0        = .3; % cm
xp0       = 0.0;
yp0       = 0.0;

% Simulation parameters
stepsize = 0.005; % cm
distance = 32; % cm

% optimization parameters
iterations = 20;
tolerance  = 1e-6; 

%-- Lattice description
elements ='QDQ'; % element types
location = [8,16,24]; % element location
lengths = [3.6400,3.8500,3.6400]; % element length
str = [220,15.7000,-220]; % quadrupole strength (kappa)
bend_ang = [0 10 0]*pi/180; % bend angle
invrho = bend_ang./lengths; % inverse rho
opt = [0,0,0];

% -- assign params to menv structures
clm.maketarget([1,1,1,1],[1,1,1,1])
clm.makeoptiset(iterations,tolerance)  % Makes file optiset
clm.defmatcher() % loads matcher settings

ic = struct();
ic.x0 = x0; ic.y0=y0; ic.xp0=xp0; ic.yp0=yp0; ic.D0 =0 ; ic.Dp0 = 0;
clm.makeparams(emittance,perveance,ic,stepsize,distance) % makes param file
clm.defparam() % loads params

clm.deflattice(elements,location,lengths,str,opt,invrho) % loads lattice

% -- look at initial solution
clm.solve()

% -- solve for periodic solution
clm.periodicmatcher()


\end{lstlisting}


\end{document}